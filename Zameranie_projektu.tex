
\documentclass[12pt,twoside,slovak,a4paper]{article}

\usepackage[slovak]{babel}
%\usepackage[T1]{fontenc}
\usepackage[IL2]{fontenc} % lepšia sadzba písmena Ľ než v T1
\usepackage[utf8]{inputenc}
\usepackage{graphicx}
\usepackage{url} % príkaz \url na formátovanie URL
\usepackage{hyperref} % odkazy v texte budú aktívne (pri niektorých triedach dokumentov spôsobuje posun textu)
\usepackage[left=2.5cm, right=2.5cm, top=3cm, bottom=3cm]{geometry}
\linespread{1.3}
\usepackage{cite}
%\usepackage{times}


\title{\textbf{Rozvoj Udržateľnej Platformy Pre Prenájom}} % 
\author{Andrii Dosyn, Anatolii Fediun, Maksym Fizer}
\date{\small 12. October 2025} % upravte

\begin{document}

\maketitle

\section*{Akronym projektu}
\textbf{R.U.P.P }


\section*{Anotácia projektu}

Projekt RUPP sa zaoberá vytvorením peer-to-peer platformy pre zdieľanie a prenájom vecí na slovenskom trhu. Cieľom je analyzovať uskutočniteľnost digitálnej platformy, ktorá umožní ľuďom prenajímať si navzájom rôzne predmety – od náradia a techniky až po vybavenie na podujatia či športové potreby. 

Základ projektu tvorí komplexná analýza trhu, ktorá skúma dopyt po takýchto službách na Slovensku, identifikuje kľúčové kategórie prenajímaných vecí a mapuje konkurenčné prostredie. Okrem finančného modelu a návratnosti investície projekt hodnotí aj technologické riešenia, právne aspekty prenájmu medzi súkromnými osobami a mechanizmy budovania dôvery v komunite používateľov.

Práca na projekte bude mať tri hlavné časti. Najprv sa pozrieme na úspešné zahraničné platformy, aby sme identifikovali osvedčené postupy a potenciálne úskalia. Následne vykonáme prieskum trhu na Slovensku, aby sme zistili, aké kategórie vecí majú ľudia záujem požičiavať a prenajímať. Nakoniec navrhneme konkrétny podnikateľský model a technickú architektúru platformy prispôsobenú slovenským podmienkam. 

Výsledkom nebude len teoretická analýza, ale praktický plán pre spustenie platformy. Ten pomôže lepšie pochopiť potenciál zdieľanej ekonomiky na Slovensku a vytvorí základ pre skutočné riešenie, ktoré prinesie úžitok majiteľom nevyužitých vecí, používateľom hľadajúcim lacnejšie alternatívy ku kúpe, a zároveň prispeje k udržateľnejšej spotrebe a zníženiu odpadu.

vysledok


\end{document}